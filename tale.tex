%\PassOptionsToPackae{gray}{xcolor}
\documentclass[hyperref={pdfpagelabels=false},12pt]{beamer}
\setbeamertemplate{frametitle}[default][center]
\mode<presentation>
{
 \usetheme{Warsaw}      % or try Darmstadt, Madrid, Warsaw, ...
 \usecolortheme{default} % or try albatross, beaver, crane, ...
 \usefonttheme{default}  % or try serif, structurebold, ...
 \setbeamertemplate{footline}[frame number]
 \setbeamertemplate{caption}[numbered]
}

\usepackage[utf8]{inputenc}
\usepackage{mathtools}
\usepackage{bm}
\usepackage{helvet}
\usepackage{listings}
\usepackage{gensymb}
\usepackage{array}
\usepackage{times}
\usepackage{xcolor}
\usepackage{default}
\usepackage{ulem}
\usepackage{minted}
\usepackage{hyperref}
\usepackage{booktabs}

% Great Commands
\newcommand{\ig}[2]{\includegraphics[width=#1\linewidth]{#2}}
\newcommand{\mybutton}[2]{\hyperlink{#1}{\beamerbutton{{#2}}}}
\newcommand{\myvbutton}[2]{\vfill\hyperlink{#1}{\beamerbutton{{#2}}}}
\newcommand{\code}[2]{\mintinline{#1}{#2}}
\newcommand{\haskell}[1]{\mintinline{haskell}{#1}}
\newcommand{\sql}[1]{\mintinline{sql}{#1}}
\newcommand{\unnamedUrl}[1]{\href{#1}{\color{blue}{#1}}}
\newcommand{\namedUrl}[2]{\href{#1}{\color{blue}{#2}}}
\newcommand{\pygment}[3]{\inputminted[bgcolor=lightgray,linenos,fontsize=#1]{#2}{#3}}
\newcommand{\pygmentLines}[5]{\inputminted[bgcolor=lightgray,linenos,fontsize=#1,firstline=#2,lastline=#3,autogobble]{#4}{#5}}

% Color Scheme
\definecolor{pittblue}{RGB}{28,41,87}
\definecolor{pittgold}{RGB}{205,184,125}
\setbeamercolor{structure}{fg=pittgold}
\setbeamercolor{button}{bg=pittblue}

\title[CRUD]{{An Introduction to Type-Safe Database Libraries in Haskell}}
\author[CRUD]{{Barry Moore II}}
\institute[CRC]{Center for Research Computing \\ University of Pittsburgh}
\date{}

\beamertemplatenavigationsymbolsempty

\begin{document}

\begin{frame}[plain]
\titlepage
\end{frame}

\begin{frame}{Outline}
  \begin{itemize}
    \item Overview of Database Libraries in Haskell
    \item Basic CRUD Backends In:
    \begin{itemize}
      \item Esqueleto/Persistent
      \item Beam
      \item Opaleye
    \end{itemize}
  \end{itemize}
\end{frame}

% \pygment{\normalsize}{haskell}{code/Functor.hs}
% \begin{itemize}
%   \item The function \haskell{(a -> b)} is ``lifted'' into the structure \haskell{f}
% \end{itemize}  

\begin{frame}{Disclaimers}
  \begin{itemize}
    \item I am not a ``database person''
    \item My previous databases were all MongoDB manipulated with PyMongo
    \item The databases I ``maintain'' could probably be simple files
    \item Motivation for this talk: \unnamedUrl{https://williamyaoh.com/posts/2019-12-14-typesafe-db-libraries.html}
    \item Code is available: \unnamedUrl{https://github.com/barrymoo/haskell-tale-of-crud-apps}
  \end{itemize}
\end{frame}

\begin{frame}{Database Libraries in Haskell}
  \begin{itemize}
    \item Embed SQL in Haskell (the ``simple'' packages)
    \begin{itemize}
      \item SQLite: \unnamedUrl{https://hackage.haskell.org/package/sqlite-simple}
      \item PostgreSQL: \unnamedUrl{https://hackage.haskell.org/package/postgresql-simple}
      \item MySQL: \unnamedUrl{https://hackage.haskell.org/package/mysql-simple}
    \end{itemize}
    \item Why would you embed SQL queries into a type-safe language?
    \begin{itemize}
      \item Persistent+Esqueleto: Backend agnostic, hides boilerplate behind TH
      \item Beam: Backend agnostic with boat loads of boilerplate
      \item Opaleye: PostgreSQL specific, heavy use of arrows
    \end{itemize}
  \end{itemize}
\end{frame}

\begin{frame}{What are we storing in the database?}
  \pygment{\scriptsize}{haskell}{code/Person.hs}
\end{frame}

\begin{frame}{Persistent+Esqueleto}
  \begin{itemize}
    \item Persistent
    \begin{itemize}
        \item type-safe, backend agnostic serialization layer
        \item relies on template haskell to avoid boilerplate
        \item backends are available via \haskell{pesistent-*} libraries 
    \end{itemize}
    \item Esqueleto
    \begin{itemize}
        \item type-safe, eDSL for SQL queries
        \item specific to Persistent
    \end{itemize}
    \item Let's jump in!
  \end{itemize}
\end{frame}


\begin{frame}{Persistent+Esqueleto: Type Definitions}
  \pygmentLines{\scriptsize}{15}{18}{haskell}{code/pers-esql/src/Person.hs}
  \vspace{-1.0cm}
  \pygmentLines{\scriptsize}{20}{37}{haskell}{code/pers-esql/src/Person.hs}
\end{frame}

\begin{frame}{Persistent+Esqueleto: Migrations}
  \pygmentLines{\scriptsize}{48}{49}{haskell}{code/pers-esql/src/Person.hs}
  \pygment{\scriptsize}{sql}{code/sql/createTable.sql}
\end{frame}

\begin{frame}{Persistent+Esqueleto: Create/Insert}
  \pygmentLines{\scriptsize}{51}{57}{haskell}{code/pers-esql/src/Person.hs}
  \pygment{\scriptsize}{sql}{code/sql/insertInto.sql}
\end{frame}

\begin{frame}{Persistent+Esqueleto: Read/Get}
  \pygmentLines{\scriptsize}{59}{67}{haskell}{code/pers-esql/src/Person.hs}
  \pygment{\scriptsize}{sql}{code/sql/selectWhere.sql}
\end{frame}

\begin{frame}{Persistent+Esqueleto: Update}
  \pygmentLines{\scriptsize}{69}{77}{haskell}{code/pers-esql/src/Person.hs}
  \pygment{\scriptsize}{sql}{code/sql/updateSetWhere.sql}
\end{frame}

\begin{frame}{Persistent+Esqueleto: Delete}
  \pygmentLines{\scriptsize}{79}{87}{haskell}{code/pers-esql/src/Person.hs}
  \pygment{\scriptsize}{sql}{code/sql/deleteWhere.sql}
\end{frame}

\begin{frame}{Persistent+Esqueleto: Fresh Test Database}
  \pygmentLines{\scriptsize}{19}{24}{haskell}{code/pers-esql/test/Spec.hs}
\end{frame}

\begin{frame}{Persistent+Esqueleto: Test Insert}
  \pygmentLines{\scriptsize}{28}{31}{haskell}{code/pers-esql/test/Spec.hs}
  \pygmentLines{\scriptsize}{40}{46}{haskell}{code/pers-esql/test/Spec.hs}
\end{frame}

\begin{frame}{Persistent+Esqueleto: Takeaways}
  \begin{itemize}
    \item I don't like defining the datatypes inside Template Haskell, because
    \begin{itemize}
      \item The generated data constructors are annoying (e.g.
        \haskell{usernameUsername})
      \item The capitalization of fields matters (uniqueness)
    \end{itemize}
    \item Aside from my gripes it was pretty easy to use
    \item I will likely use this to wrap an existing MariaDB database in the future
  \end{itemize}
\end{frame}

\begin{frame}{Beam}
  Code here...
\end{frame}

\begin{frame}{Opaleye}
  Code here...
\end{frame}

\end{document}
